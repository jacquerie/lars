\documentclass[conference,compsoc]{IEEEtran}

\ifCLASSOPTIONcompsoc
  \usepackage[nocompress]{cite}
\else
  \usepackage{cite}
\fi

\ifCLASSINFOpdf
\else
\fi

\usepackage{hyperref}
\usepackage[utf8]{inputenc}

% correct bad hyphenation here
\hyphenation{}

% See: http://tex.stackexchange.com/questions/65671/ieeetran-authorrefmark-and-affiliation
\DeclareRobustCommand*{\IEEEauthorrefmark}[1]{%
    \raisebox{0pt}[0pt][0pt]{\textsuperscript{\footnotesize\ensuremath{#1}}}}

\begin{document}

\title{Tibor: Event-Driven Static File Cache and Use in an Erlang File Server}
\author{
  \IEEEauthorblockN{
    Daniel Lovasko\IEEEauthorrefmark{1},
    Jacopo Notarstefano\IEEEauthorrefmark{2},
    Laura Rueda\IEEEauthorrefmark{2} and
    Jan Stypka\IEEEauthorrefmark{2}
  }
  \IEEEauthorblockA{
    \IEEEauthorrefmark{1}
    lovasko@freebsd.org (The FreeBSD Project)
  }
  \IEEEauthorblockA{
    \IEEEauthorrefmark{2}
    \{jacopo.notarstefano, laura.rueda, jan.stypka\}@cern.ch (CERN)
  }
}

\maketitle

\begin{abstract}
In this paper we introduce Tibor, an innovative implementation of an
event-driven server-side cache for static files. Tibor was developed during the
\href{https://facebook.com/hackepfl}{EPFL Facebook Hackathon}, in the wee hours
of the morning. The authors take full responsibility for the horrors contained
in its codebase.
\end{abstract}

\IEEEpeerreviewmaketitle

\section{Introduction}
Drawing on their experience as contributors to Invenio, the library powering
several of the biggest digital libraries of Europe\footnote{Among which the
CERN Document Server, INSPIRE and the Infoscience project at EPFL}, the
authors noticed the opportunity for creating a software specialized in the
server-side caching of static files.

In fact, the serving of static files of sizes ranging from modest (for example,
the papers of the High Energy Physics community stored in INSPIRE) to
significant (for example, the historical movies stored in the CERN Document
Server) is at odds with the aging architecture of the Invenio library, a
problem that can only be alleviated, but not solved, by falling back to the
Apache Web Server.

The authors propose Tibor, a novel solution written in C as a stand-alone
library with minimal dependencies. To demonstrate its versatility, the authors
provide a proof-of-concept integration with Elli, an Erlang web server.

Even if this won't provide a solution to the problem the authors are facing, we
confide that this integration constitutes sufficient proof of the wide
applicability of this solution to disparate domains\footnote{Possibly including
the problem of visiting California, or at least getting a few t-shirts.}.

\hfill DL, JN, LR, and JS
 
\hfill April 19, 2015

\section{High Level Architecture}
As most caches, Tibor in structured in two levels, conventionally called ``L1''
and ``L2''. It is expected that accesses to the L1 cache are significantly
faster than those to the second level, at the price of a smaller cache size.
In the next section we will describe the data structures and the search
algorithms that Tibor uses to achieve this goal.

Tibor is configured, as nowadays is common, by editing a file in the JSON
format. In the third section we will describe its format, the currently
supported features and the plans for extending them.

In the fourth section we will describe Jiří, the mechanism that Tibor uses to
communicate with the Elli web server. As it will be clear from the discussion,
communicating between BEAM, the Erlang Virtual Machine, and \texttt{tibord},
the daemonized version of the Tibor library, requires a little ingenuity and
some knowledge of the inner working of these two pieces of software.

Finally, in the last section, we will describe the Simko prototype to visualize
and provide guidance in improving Tibor's performance.

\section{Data Structures}
TODO

\section{Configuration}
We first present a complete example of a Tibor's configuration file, and then
we will comment on its features.

TODO

We remark a feature afforded by Tibor's unique architecture: it is possible to
invalidate a single cached element just by changing its permissions, provided
that the appropriate policy was selected during configuration.

The authors recognize that configuring a large cache could prove to be tedious.
Therefore, they are planning to include the possibility of defining \emph{memory pools}
described by regular expressions, as in the following example:

TODO

\section{Internals}
TODO

\section{Simko: a visualization layer for Tibor}
While developing Tibor, the authors noticed that its users could benefit from
inspecting the status of the cache and diagnosing its inefficiencies, either
caused by misconfiguration or by bugs in the software. Therefore the authors
decided to implement a graphical tool that would allow the users in easily
identifying missed opportunities for optimization and wasted resources.

Since Tibor's support for logging is still in its infancy, the communication
between Simko and Tibor is achieved by reading and writing shared files in the
CSV format. We auspicate that in the future a more robust medium of
communication is estabilished between these two softwares.

Despite this limitation, Simko can show the current status of the cache,
automatically refreshing every 5 seconds. The colors and the lenghts of the
bars encode the number of times an item was accessed and found in L1, L2 or not
found at all.

The user can immediately see which files where accessed often despite not being
picked for the cache. On the other hand, she can also see which files she
selected weren't accessed as much as she expected.

Simko goes even further by automatically suggesting which files should be
evicted by the user from the cache and which files should be included instead.

\section{Conclusion}
In this paper we presented Tibor, an event-driven server-side cache for static
files. TODO

\ifCLASSOPTIONcompsoc
  \section*{Acknowledgments}
\else
  \section*{Acknowledgment}
\fi

The authors would like to thank their bosses Tibor Šimko, Jiří Kunčar and Lars
Holm Nielsen for being constant sources of inspiration. The authors also thank
Marios Kogias for the productive conversations that shaped the design of the
Tibor library, and for having the courage to step back at a crucial moment in
Tibor's history.

\end{document}
